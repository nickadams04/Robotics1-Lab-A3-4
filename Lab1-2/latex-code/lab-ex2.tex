% For XeLaTeX or LuaTex compilers
% to use minted copy vs_code_settings.json to your vscode settings

\documentclass[11pt]{article}

\usepackage[a4paper,top=3cm,bottom=3cm,left=2cm,right=2cm,marginparwidth=1.75cm]{geometry}
\setlength{\parskip}{1em}
\setlength{\headheight}{13.59999pt}
\usepackage{setspace}
\setstretch{1.15} % similar to MS Word 1.0 line spacing
% Multilanguage support
\usepackage{polyglossia}

% Math symbols
\usepackage{amsmath}
\usepackage{amssymb}

% float
\usepackage{float}

%headers
\usepackage{fancyhdr}
\pagestyle{fancy}
\lhead{Ρομποτική 1}
\chead{Άσκηση 2 - \emph{Pendubot} - A3-4}
\rhead{Χειμερινό Εξάμηνο 2025-26}

%table
\usepackage{tabularx,booktabs}
\usepackage{xcolor,colortbl}
\usepackage{multirow}
\newcolumntype{Y}{>{\centering\arraybackslash}X}
\newcolumntype{M}{>{\columncolor{blue}\centering\arraybackslash}c}

% Image figures
\usepackage{graphicx}
% Look in local images and sibling diagrams folder
\graphicspath{{./images/}{../Pendubot - Διαγράμματα/}}

% Figures inside figures
\usepackage{subfigure}

% Color text
\usepackage[dvipsnames, svgnames]{xcolor}

% Text positioning
\usepackage{ragged2e}

% code
\usepackage{listings}

% inline fractions
\usepackage{nicefrac}

% fucking arrows
\usepackage{tikz}
\usetikzlibrary{tikzmark}

% Links
\usepackage[pdfusetitle]{hyperref}
\hypersetup{
	colorlinks=true,
	linkcolor=black,
	urlcolor=teal
}



% Code snippets
% \usepackage[newfloat]{minted}

% Better figure captions
\usepackage{caption}

% Whatever
\usepackage{enumitem}

% whatever% 
\usepackage{svg}

% Fonts - For greek use a font with the required characters
\usepackage{fontspec}
\defaultfontfeatures{Ligatures=TeX}
\setmainfont{Calibri}
\newfontfamily\greekfonttt{Consolas}
\setmonofont{Consolas}
\newfontfamily\tahoma{tahoma.ttf}[
	BoldFont= tahomabd.ttf,
	Scale=0.9
]
\newfontfamily\cambria{Cambria}
\newfontfamily\consolas{Consolas}
\newfontfamily\palatino{Palatino Linotype}
\newfontfamily\trebuc{trebuc.ttf}[
	BoldFont=trebucbd.ttf,
	ItalicFont=trebucit.ttf,
	BoldItalicFont=trebucbi.ttf
]

\setdefaultlanguage{greek}

\newcounter{my_exercise_counter}
\newenvironment*{exercise}{
	\refstepcounter{my_exercise_counter}\pdfbookmark[0]{Άσκηση~\arabic{my_exercise_counter}}{ex.\arabic{my_exercise_counter}.0}
	{\rule{\textwidth}{0.1pt}\\\vspace{0.5\baselineskip}\noindent\centering\Large\bfseries Άσκηση~\arabic{my_exercise_counter}.\\
	}\vspace{0.3\baselineskip}\justifying 
}{~\vspace{18pt}}



\begin{document}
% \tahoma
\justifying

\begin{titlepage}
    % \vspace*{1cm}
    \begin{center}
        \large
        \textbf{Εθνικό Μετσόβιο Πολυτεχνείο}

        \large
        \textbf{Σχολή Ηλεκτρολόγων Μηχανικών και Μηχανικών Υπολογιστών}

        \vspace*{5cm}

		\huge
		\textbf{Ρομποτική 1}

        \Huge
        Εργαστηριακή Άσκηση 2

        \LARGE
        Pendubot

        \vspace*{9cm}

        \large
        \textbf{Ομάδα Α3-4}
        \begin{tabular}{|l | l}
            Νικόλαος Αδαμόπουλος & 03122074 \\
            Λεωνίδας Λαγομιτζής & 03121061 \\
            Ηλίας-Στυλιανός Ηλιάδης & 03122140 \\
			Ηλίας Πουλόπουλος & 03122 \\
			?? & ??
        \end{tabular}
        
    \end{center}
\end{titlepage}

\tableofcontents

\newpage

\section{Εισαγωγή}

Στην εργαστηριακή άσκηση, επιδιώξαμε να ελέγξουμε τον ρομποτικό μηχανισμό Pendubot με χρήση γραμμικού PD ελέγχου και χρήση μη γραμμικού ελέγχου.

Αρχικά, επιχειρήσαμε να τοποθετήσουμε μία μεμονωμένη άρθρωση του μηχανισμού αυτού σε γωνία $30°$ με χρήση γραμμικού PD ελέγχου, δοκιμάζοντας διάφορες τιμές των συντελεστών $K_p, K_d$. Σε όλα τα πειράματά μας, παρατηρήσαμε ωστόσο σημαντικό σφάλμα μόνιμης κατάστασης. 

Για να λάβουμε καλύτερα αποτελέσματα, περιορίσαμε την επίδραση της βαρύτητας προσθέτοντας τον όρο αντιστάθμισης $g_1$ στο καλύτερο μοντέλο μας, οπότε και αποκτήσαμε αρνητικό σφάλμα. Προκειμένου να κατανοήσουμε την επίδραση του όρου $g_1$ στο Pendubot, μελετήσαμε την ειδική περίπτωση $(K_p,K_d)=(0,0)$, διαπιστώνοντας πως ο όρος $g_1$ αδυνατεί να αντισταθμίσει πλήρως τη βαρύτητα.

Τέλος, εξετάσαμε την λειτουργία ενός πλήρους µη γραμμικού ελεγκτή (δύο φάσεων) για την εξισορρόπηση του αναστροφου εκκρεμούς στην ενδιάμεση θέση ασταθούς ισορροπίας.

\section{Πειραματική Διάταξη}

Το Pendubot είναι ένας ρομποτικός μηχανισμός τύπου "ανάστροφο εκκρεμές" και έχει δύο βαθμούς ελευθερίας. Ειδικότερα, ο μηχανισμός αποτελείται από δύο κινούμενους συνδέσμους (link-1, link-2) και από δύο στροφικές αρθρώσεις  (joint-1, joint-2). Σε κάθε άρθρωση, διαθέτουμε έναν οπτικό κωδικοποιητή (encoder) για την μέτρηση της γωνίας στροφής, ενώ η πρώτη άρθρωση joint-1 ενεργοποιείται από έναν κινητήρα συνεχούς ρεύματος (DC Motor). Η δεύτερη άρθρωση, παρότι ελεύθερη να κινηθεί περιστροφικά, δεν διαθέτει ανεξάρτητο ενεργοποιητή, οπότε συνιστά μία υποοδηγούμενη άρθρωση.

Η πειραματική διάταξη αναπαριστάται στο Σχήμα 1 του εργαστηριακού οδηγού, το οποίο επισυνάπτουμε ακολούθως:

\begin{figure}[H]
	\centering
	\includegraphics[width=0.4\textwidth]{pendubot-schematic.png}
	\caption{Κινηματική Δομή ρομποτικού μηχανισμού \emph{Pendubot}}
	\label{fig:pendubot-schematic}
\end{figure}

\section{Θεωρητικό Υπόβαθρο}
\subsection*{PD Έλεγχος}

Ο PD (Proportional–Derivative) έλεγχος είναι ένας γραμμικός τρόπος ελέγχου. Η συνάρτηση του ελέγχου δίνεται από τη σχέση $u(t)=K_p e(t)+K_d  \frac{de}{dt}$ και έχει συνάρτηση μεταφοράς $G_{PD} (s)=K_p+K_d s$ και φαίνεται σχηματικά σε εικόνα της παρουσίασης της εργαστηριακής άσκησης την οποία επισυνάπτουμε ακολούθως:

\begin{figure}[H]
	\centering
	\includegraphics[width=0.4\textwidth]{pd-schematic.png}
	\caption{Block διάγραμμα ελεγκτή PD}
	\label{fig:pd-schematic}
\end{figure}

\subsection*{Χαρακτηριστικά Απόκρισης}

\begin{itemize}
\item 
{
\textbf{Σφάλμα Μόνιμης Κατάστασης}

Στα πειράματά μας, θα εργαστούμε με τη βηματική είσοδο $u(t)=1, t\geq0$ και $u(t)=0 , t<0$. Αν το σφάλμα παίρνει τελική τιμή, δηλαδή αν υπάρχει το όριο $\lim_{t \to \infty}(e(t))$, το σφάλμα μόνιμης κατάστασης δίνεται από τoν τύπο $e_{ss}=1/(1+K_P )$, όπου $K_P=\lim_{s \to 0}(G(s))$ και $G(s)$ η συνάρτηση , μεταφοράς του συστήματος ανοιχτού βρόγχου σε μετασχηματισμό Laplace.
}
\item 
{
\textbf{Υπερύψωση}

Η υπερύψωση (overshoot) εκφράζει σε ποσοστό το κατά πόσο η μέγιστη τιμή της απόκρισης ξεπερνά την τιμή μόνιμης κατάστασης. Έτσι, αν θεωρήσουμε απόκριση $e$, η υπερύψωση ορίζεται ως 

\begin{equation*}
	M_p=\frac{e_{max}-e_{ss}}{e_{ss}}100\%
\end{equation*}

όπου $e_{ss}$ η τιμή τελικής κατάστασης.
}
\item 
{
\textbf{Χρόνος Ανύψωσης}

Ο χρόνος ανύψωσης (rise time) ορίζεται ως το χρονικό διάστημα που μεσολάβησε προκειμένου η απόκριση να μεταβεί από το $10\%$ της τελικής τιμής στο $90\%$ της τελικής τιμής.
}
\end{itemize}

\subsection*{Αντιστάθμιση Βαρύτητας}
Σύμφωνα με παράρτημα του εργαστηριακού οδηγού, οι εξισώσεις κίνησης του Pendubot είναι οι ακόλουθες:

\begin{gather*}
	\tau_1 = d_{11} \frac{d^2 q_1}{dt^2} + d_{12} \frac{d^2 q_2}{dt^2} + c_{11} \dot{q_1} + \varphi_1 \\
	0 = d_{21} \frac{d^2 q_1}{dt^2} + d_{22} \frac{d^2 q_2}{dt^2} + c_{21} \dot{q_1} + \varphi_2 \\
\end{gather*}

Όπου, $q=\begin{bmatrix} q_1 & q_2 \end{bmatrix}^T$, το διάνυσμα των γωνιών των συνδέσμων 1 και 2 αντίστοιχα,  $\tau=\begin{bmatrix} \tau_1 & 0 \end{bmatrix}^T$, το διάνυσμα της ροπής που εφαρμόζεται στους δύο συνδέσμους, $D(q)=\begin{bmatrix} d_{11} & d_{12} &;& d_{21} & d_{22} \end{bmatrix}$ η μήτρα αδράνειας του ρομπότ, $C(q)=\begin{bmatrix} c_{11} & c_{12} &;& c_{21} & c_{22} \end{bmatrix}$ οι όροι Coriolis και φυγόκεντρου δυνάμεως και $g(q)=\begin{bmatrix} \varphi_1 & \varphi_2 \end{bmatrix}^T$ το διάνυσμα της βαρύτητας.
Οι εξισώσεις κίνησης προέκυψαν με βάση το Δυναμικό Μοντέλο Lagrange Ρομποτικού Χειριστή, σύμφωνα με το οποίο: $\tau=D(q)\ \frac{d^2q}{dt^2}+C(\dot{q},q)\dot{q}+g(q)$.
Θεωρώντας τοπικό νόμο ελέγχου $(\dot{q}=0): u=K_p e(t)+K_d\dot{\ e(t)}+g(q)$ και μελετώντας την ευστάθεια κατά Lyapunov παίρνουμε:

\begin{equation*}
	V(q,\dot{q})=\frac{1}{2}(e^T K_p e+{(\dot{e})}^TD(q)\dot{e})
\end{equation*}

Παρατηρούμε ότι $V(q,\dot{q})=0$, όταν $e=0\ και \dot{e}=0$. Παραγωγίζοντας την συνάρτηση Lyapunov παίρνουμε

\begin{equation*}
\dot{V}(q,\dot{q})=e^TK_p\dot{e}+{(\dot{e})}^TD(q)\frac{d^2e}{dt^2}+\frac{1}{2}{(\dot{e})}^T(\dot{D})(\dot{e})\end{equation*}

Δεδομένου ότι $e=q_d-q$ μπορούμε να γράψουμε:

\begin{equation*}\dot{V}(q,\dot{q})=-e^TK_p\dot{q}+{(\dot{q})}^TD(q)\frac{d^2q}{dt^2}+\frac{1}{2}{(\dot{q})}^T(\dot{D})(\dot{q})\end{equation*}

Επειδή η μήτρα $\dot{D}-2C$ είναι αντισυμμετρική και ισχύει $w^T(\dot{D}-2C)w=0,\ \forall w \in \mathbb{R}^n$, παίρνουμε:

\begin{gather*}
\dot{V}(q,\dot{q})=-e^TK_p\dot{q}+{(\dot{q})}^T(\tau-C\dot{q}-g(q))+\frac{1}{2}{(\dot{q})}^T(\dot{D})(\dot{q}) \\
\dot{V}(q,\dot{q})=-e^TK_p\dot{q}+{(\dot{q})}^T(\tau-g(q)) \\
\dot{V}(q,\dot{q})=-e^TK_p\dot{q}\ +{(\dot{q})}^T(K_pe+K_d\dot{e}+g(q)-g(q)) \\
\dot{V}(q,\dot{q})=-e^TK_p\dot{q}\ +{(\dot{q})}^T(K_pe+K_d\dot{e}) \\
\dot{V}(q,\dot{q})={(\dot{q})}^TK_d\dot{e} \\
\dot{V}(q,\dot{q})=-{(\dot{q})}^TK_d\dot{q}\le0
\end{gather*}

Επιπλέον, έχουμε $\dot{V}(q,\dot{q})=0$ όταν $|\dot{q}|=0$, οπότε και για $\dot{q}=0$ έχουμε ευστάθεια Lyapunov.
Άρα λοιπόν, αν θεωρήσουμε τοπικό νόμο ελέγχου $u=K_pe(t)+K_d\dot{\ e(t)}+g(q)$, αναμένουμε το σύστημα να είναι ευσταθές.

\subsection*{Μη Γραμμικός Ελεγκτής Δύο Φάσεων}

Στον μη γραμμικό ελεγκτή δύο φάσεων πρέπει να ελέγξουμε την φάση της ταλάντωσης και την φάση της εξισορρόπησης του link-2 σε όρθια θέση, οπότε χρειαζόμαστε δύο ελεγκτές.


\pagebreak

\section{Πειραματική Ρύθμιση PD Ελεγκτή}

Στο πρώτο μέρος του πειράματος πραγματοποιήσαμε \textbf{σάρωση παραμέτρων} (grid search) του PD ελεγκτή σε διακριτό, μη ομοιόμορφο πλέγμα τιμών: επιλέξαμε $4$ διακριτές τιμές για τον $K_p$ και, για καθεμία, ένα διαφορετικό σύνολο $3$ τιμών για τον $K_d$ (σύνολο $12$ συνδυασμοί), επί των οποίων εφαρμόσαμε βηματική εντολή θέσης στην άρθρωση $q_1$ (από τη φυσική θέση σε $\theta_d = 30^{\circ}$).

Για κάθε δοκιμή καταγράψαμε:
\begin{itemize}[nosep]
	\item το \textbf{σφάλμα μόνιμης κατάστασης} $e_{ss}$,
	\item τον \textbf{χρόνο ανύψωσης} $t_r$ (10\%–90\%),
	\item το \textbf{ποσοστό υπερύψωσης} $M_p$.
\end{itemize}

Τα αποτελέσματα συνοψίζονται ενδεικτικά στον παρακάτω πίνακα, όπου οι τιμές θα συμπληρωθούν μετρητικά:

\begin{table}[H]
    \centering
    \caption{Πειραματικά αποτελέσματα ρύθμισης PD για διάφορα ζεύγη $(K_p,K_d)$}
    \begin{tabular}{|c| c c | c c c|}
         \toprule
            Δοκιμή & $K_p$ & $K_d$ & $e_{ss}$ (deg) & $t_r$ (s) & $M_p$ (rad) \\
            \midrule
            1  & \multirow{3}{*}{$1$}  & $0$    & $19.34$ & $0.18$ & $0.1$    \\
            2  &                       & $0.15$ & $19.2$  & $0.24$ & $0.014$  \\
            3  &                       & $0.23$ & $19.48$ & $0.48$ & $0.005$  \\ \midrule
            4  & \multirow{3}{*}{$5$}  & $0.25$ & $8.04$  & $0.12$ & $0.11$   \\
            5  &                       & $0.4$  & $8.33$  & $0.18$ & $0.031$  \\
            6  &                       & $0.48$ & $7.68$  & $0.48$ & $0.0062$ \\ \midrule
            7  & \multirow{3}{*}{$10$} & $0.55$ & $4.73$  & $0.12$ & $0.038$  \\
            8  &                       & $0.65$ & $4.58$  & $0.36$ & $0.014$  \\
            9  &                       & $0.7$  & $4.73$  & $0.3$  & $0.0088$  \\ \midrule
            10 & \multirow{3}{*}{$20$} & $0.8$  & $2.64$  & $0.12$ & $0.043$  \\
            11 &                       & $0.95$ & $2.136$ & $0.42$ & $0.0038$ \\
            12 &                       & $1.3$  & $2.28$  & $0.42$ & $0.0038$ \\
            \bottomrule
    \end{tabular}
\end{table}

Για καθεμία από τις παραπάνω δοκιμές, παραθέτουμε και τα αντίστοιχα διαγράμματα, τα οποία συλλέχθηκαν κατά την εκτέλεση.


% Grid of PD trial figures: rows by K_p, columns by K_d
\begin{figure}[H]
\centering
\renewcommand{\thesubfigure}{\arabic{subfigure}.}

% K_p = 1
\subfigure[$K_p=1,\ K_d=0.00$]{\includegraphics[width=.30\textwidth]{1_0.00.pdf}}
\subfigure[$K_p=1,\ K_d=0.15$]{\includegraphics[width=.30\textwidth]{1_0.15.pdf}}
\subfigure[$K_p=1,\ K_d=0.23$]{\includegraphics[width=.30\textwidth]{1_0.23.pdf}}

% K_p = 5
\subfigure[$K_p=5,\ K_d=0.25$]{\includegraphics[width=.30\textwidth]{5_0.25.pdf}}
\subfigure[$K_p=5,\ K_d=0.40$]{\includegraphics[width=.30\textwidth]{5_0.40.pdf}}
\subfigure[$K_p=5,\ K_d=0.48$]{\includegraphics[width=.30\textwidth]{5_0.48.pdf}}

% K_p = 10
\subfigure[$K_p=10,\ K_d=0.55$]{\includegraphics[width=.30\textwidth]{10_0.55.pdf}}
\subfigure[$K_p=10,\ K_d=0.65$]{\includegraphics[width=.30\textwidth]{10_0.65.pdf}}
\subfigure[$K_p=10,\ K_d=0.70$]{\includegraphics[width=.30\textwidth]{10_0.70.pdf}}

% K_p = 20
\subfigure[$K_p=20,\ K_d=0.80$]{\includegraphics[width=.30\textwidth]{20_0.80.pdf}}
\subfigure[$K_p=20,\ K_d=0.95$]{\includegraphics[width=.30\textwidth]{20_0.95.pdf}}
\subfigure[$K_p=20,\ K_d=1.30$]{\includegraphics[width=.30\textwidth]{20_1.30.pdf}}

\caption{Πλέγμα δοκιμών PD: κάθε σειρά αντιστοιχεί σε σταθερό $K_p$ (1, 5, 10, 20) και κάθε στήλη σε διαφορετικό $K_d$.}
\label{fig:pd-trials-grid}
\end{figure}

Παρατηρούμε ότι η αύξηση του \(K_p\) βελτιώνει το σφάλμα μόνιμης κατάστασης, (βλέπουμε και στις μετρήσεις ότι μειώνεται από \(19.34 \deg \text{σε} 2.28 \deg\)), όμως δεν καταφέρνει να το μηδενίσει. Αυτό ήταν αναμενόμενο, καθώς δεν υπάρχει κάποια αντιστάθμιση της βαρύτητας, όπως θα δούμε στο επόμενο κομμάτι. 

Παράλληλα η αύξηση του όρου διαφόρισης \(K_d\) προκαλεί αφενός ελαφρά αύξηση του χρόνου ανύψωσης, αφετέρου ρίχνει σημαντικά το ποσοστό υπερύψωσης. Εξομαλύνει έτσι τη ταλαντωτική συμπεριφορά του συστήματος, μειώνοντας από 10 μέχρι 20 φορές το ποσοστό υπερύψωσης, ενώ ο χρόνος ανύψωσης αυξάνεται μόλις 2 με 3 φορές (από την ήδη χαμηλή τιμή του). 

Η πορεία του βέλτιστου ζεύγους είναι η εξής:

\begin{figure}[H]
    \centering
    \includegraphics[width=0.48\textwidth]{best_Kpd.pdf}
    \includegraphics[width=0.48\textwidth]{ess_Kpd.pdf}
    \caption{Βέλτιστο ζεύγος \(K_d\) και αντίστοιχο \(e_{ss}\) ανά \(K_p\).}
    \label{fig:best-kpd}
\end{figure}

Βλέπουμε ότι όσο αυξάνεται ο αναλογικός όρος \(K_p\) μειώνεται η απαιτούμενη αύξηση του \(K_d\) γιά να αντιμετωπιστούν οι ταλαντώσεις του, ενώ μειώνεται και η μείωση του σφάλματος μόνιμης κατάστασης. Και οι δύο αυτές παρατηρήσεις μας οδηγούν ότι υπάρχει ιδανικό ζεύγος παραμέτρων είναι το \((20,\ 0.95)\), αφού πέραν αυτού αύξηση είτε του \(K_p\) είτε του \(K_d\) δεν μας προσφέρει κάτι παραπάνω. Στο ζεύγος αυτό θα βασιστούμε για τη βελτίωση με χρήση αντιστάθμισης.

\section{Αντιστάθμιση Βαρύτητας}



\section{Μη Γραμμικός Ελεγκτής 2 φάσεων}

Στο τελευταίο μέρος του πειράματος μελετήθηκε η λειτουργία ενός \textbf{μη-γραμμικού ελεγκτή δύο φάσεων} για την εξισορρόπηση του αναστροφου εκκρεμούς στη μεσαία θέση ασταθούς ισορροπίας (\emph{middle balancing position}). Η πειραματική διάταξη που έχουμε (το σύστημα είναι υπο-οδηγούμενο) επιτρέπει την έμμεση επιρροή της κίνησης του δεύτερου συνδέσμου μέσω των επιταχύνσεων που επιβάλλονται στον πρώτο. Ο συγκεκριμένος ελεγκτής αποτελείται από δύο φάσεις ελέγχου, τη φάση της ταλάντωσης και τη φάση της εξισορρόπησης.

Στη \textbf{φάση ταλάντωσης}, στόχος του ελεγκτή είναι το εκκρεμές να προσεγγίσει την επιθυμητή θέση ισορροπίας μέσω της αύξησης της μηχανικής ενέργειας του συστήματος. Η δυναμική εξίσωση του συστήματος έχει τη μορφή:
\[
\dot{x} = A x + b u + d(x)
\]
όπου $x$ το διάνυσμα κατάστασης, $u$ η ροπή στο πρώτο σύνδεσμο και $d(x)$ οι μη γραμμικοί όροι. Στη φάση αυτή πραγματοποιούνται ελεγχόμενες ταλαντώσεις στην πρώτη άρθρωση, μεταφέροντας ενέργεια στη δεύτερη άρθρωση. Η πρώτη φάση ολοκληρώνεται όταν το σύστημα πλησιάσει τη μεσαία θέση ισορροπίας.

Έπειτα, ενεργοποιείται η \textbf{φάση εξισορρόπησης}, όπου ο ελεγκτής βασίζεται στη μη-γραμμική συνάρτηση Lyapunov που έχει τη μορφή:
\[
V_i(x,t) = \ln\bigl(1 + x^T P x\bigr)
\]
Επιλέγεται η λογαριθμική συνάρτηση Lyapunov για να εξασφαλιστεί θετική τιμή και καλύτερη προσέγγιση, καθώς παρέχει ανοχή σε μη γραμμικές μεταβολές.

Κατά την διεξαγωγή του πειράματος η επιθυμητή μεσαία θέση επιτεύχθηκε με επιτυχία σε μικρό χρονικό διάστημα και με λιγοστές αλλαγές επιτάχυνσης από τον μη γραμμικό ελεγκτή δύο φάσεων. Επιπλέον, σε μεταβολή της θέσης μετά την ισορροπία, η φάση εξισορρόπησης εξασφαλίζει την επαναφορά στην επιθυμητή θέση. Επομένως, διαπιστώνεται επιτυχής λειτουργία του ελεγκτή.

\end{document}