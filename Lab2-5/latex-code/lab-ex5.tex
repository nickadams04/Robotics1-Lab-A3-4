% For XeLaTeX or LuaTex compilers
% to use minted copy vs_code_settings.json to your vscode settings

\documentclass[11pt]{article}

\usepackage[a4paper,top=3cm,bottom=3cm,left=2cm,right=2cm,marginparwidth=1.75cm]{geometry}
\setlength{\parskip}{1em}
\setlength{\headheight}{13.59999pt}
\usepackage{setspace}
\setstretch{1.15} % similar to MS Word 1.0 line spacing
% Multilanguage support
\usepackage{polyglossia}

% Math symbols
\usepackage{amsmath}
\usepackage{amssymb}

% float
\usepackage{float}

%headers
\usepackage{fancyhdr}
\pagestyle{fancy}
\lhead{Ρομποτική 1}
\chead{Άσκηση 2 - \emph{Pendubot} - A3-4}
\rhead{Χειμερινό Εξάμηνο 2025-26}

%table
\usepackage{tabularx,booktabs}
\usepackage{xcolor,colortbl}
\usepackage{multirow}
\newcolumntype{Y}{>{\centering\arraybackslash}X}
\newcolumntype{M}{>{\columncolor{blue}\centering\arraybackslash}c}

% Image figures
\usepackage{graphicx}
% Look in local images and sibling diagrams folder
\graphicspath{{./images/}{../Pendubot - Διαγράμματα/}}

% Figures inside figures
\usepackage{subfigure}

% Color text
\usepackage[dvipsnames, svgnames]{xcolor}

% Text positioning
\usepackage{ragged2e}

% code
\usepackage{listings}

% inline fractions
\usepackage{nicefrac}

% fucking arrows
\usepackage{tikz}
\usetikzlibrary{tikzmark}

% Links
\usepackage[pdfusetitle]{hyperref}
\hypersetup{
	colorlinks=true,
	linkcolor=black,
	urlcolor=teal
}



% Code snippets
% \usepackage[newfloat]{minted}

% Better figure captions
\usepackage{caption}

% Whatever
\usepackage{enumitem}

% whatever
% \usepackage{svg}

% Fonts - For greek use a font with the required characters
\usepackage{fontspec}
\defaultfontfeatures{Ligatures=TeX}
\setmainfont{Calibri}
\newfontfamily\greekfonttt{Consolas}
\setmonofont{Consolas}
\newfontfamily\tahoma{tahoma.ttf}[
	BoldFont= tahomabd.ttf,
	Scale=0.9
]
\newfontfamily\cambria{Cambria}
\newfontfamily\consolas{Consolas}
\newfontfamily\palatino{Palatino Linotype}
\newfontfamily\trebuc{trebuc.ttf}[
	BoldFont=trebucbd.ttf,
	ItalicFont=trebucit.ttf,
	BoldItalicFont=trebucbi.ttf
]

\setdefaultlanguage{greek}

\newcounter{my_exercise_counter}
\newenvironment*{exercise}{
	\refstepcounter{my_exercise_counter}\pdfbookmark[0]{Άσκηση~\arabic{my_exercise_counter}}{ex.\arabic{my_exercise_counter}.0}
	{\rule{\textwidth}{0.1pt}\\\vspace{0.5\baselineskip}\noindent\centering\Large\bfseries Άσκηση~\arabic{my_exercise_counter}.\\
	}\vspace{0.3\baselineskip}\justifying 
}{~\vspace{18pt}}

\renewcommand{\figurename}{Σχήμα}
\renewcommand{\tablename}{Πίνακας}

\begin{document}
% \tahoma
\justifying

\begin{titlepage}
    % \vspace*{1cm}
    \begin{center}
        \large
        \textbf{Εθνικό Μετσόβιο Πολυτεχνείο}

        \large
        \textbf{Σχολή Ηλεκτρολόγων Μηχανικών και Μηχανικών Υπολογιστών}

        \vspace*{5cm}

		\huge
		\textbf{Ρομποτική 1}

        \Huge
        Εργαστηριακή Άσκηση 5

        \LARGE
        Niryo Ned2

        \vspace*{9cm}

        \large
        \textbf{Ομάδα Α3-4}
        \begin{tabular}{|l | l}
            Νικόλαος Αδαμόπουλος & 03122074 \\
            Λεωνίδας Λαγομιτζής & 03121061 \\
            Ηλίας-Στυλιανός Ηλιάδης & 03122140 \\
			Ηλίας Πουλόπουλος & 03122160 \\
			Ανδρέας Αγγελόπουλος & 03122049
        \end{tabular}
        
    \end{center}
\end{titlepage}

\tableofcontents

\newpage

% Ηλίας Π/ Λεωνίδας
\section{Εισαγωγή}

% Ηλίας Π/ Λεωνίδας
\section{1ο Μέρος} 

% Αγνοούμενος
\section{2ο Μέρος (α) - Pick and Place}

Σε αυτό το βήμα εργαστήκαμε με το free motion του ρομποτικού βραχίονα, με σκοπό να
ορίσουμε τρεις βασικές θέσεις για καθένα από τα τρία αντικείμενα. Η πρώτη θέση αντιστοιχεί
στο σημείο όπου ξεκινά το άνοιγμα του βραχίονα, η δεύτερη στο σημείο όπου αρχίζει το
κλείσιμό του για την αρπαγή του αντικειμένου και η τρίτη στη θέση όπου ο ρομποτικός
βραχίονας αφήνει το αντικείμενο. Για τη διευκόλυνση της διαδικασίας αυτής, έχει
δημιουργηθεί ένα βοηθητικό αρχείο που επιτρέπει την αποθήκευση ενδιάμεσων θέσεων του
ρομποτικού βραχίονα, προσφέροντας μεγαλύτερη ακρίβεια και ευελιξία στον χειρισμό του.
Επιπλέον, η άσκηση παρέχει τη δυνατότητα χρήσης ενός αλγορίθμου αναγνώρισης σχημάτων,
ο οποίος επιτρέπει τον εντοπισμό αντικειμένων με κυκλική μορφή. Οι χρήστες έχουν τη
δυνατότητα να αλλάξουν τη σειρά με την οποία ο ρομποτικός βραχίονας συλλέγει τα
αντικείμενα, καθώς και να ορίσουν τα σημεία στα οποία αυτά τοποθετούνται. Τέλος,
υποστηρίζεται η πλήρως αυτοματοποιημένη εκτέλεση της διαδικασίας, γεγονός που καθιστά
την άσκηση ιδιαίτερα ευέλικτη και κατάλληλη για εκπαιδευτικούς σκοπούς και πειραματισμό
στον προγραμματισμό κινήσεων και στον χειρισμό αντικειμένων.

Η πρώτη εικόνα παρουσιάζει το κεντρικό μενού ελέγχου του ρομποτικού συστήματος Niryo,
μέσω τερματικού. Ο χρήστης μπορεί να επιλέξει μεταξύ διαφορετικών λειτουργιών, όπως
ζωντανή προβολή κάμερας, ανίχνευση κυκλικών αντικειμένων, χειροκίνητη ή
αυτοματοποιημένη διαδικασία pick and place, καθώς και έξοδο από το πρόγραμμα.

Η δεύτερη εικόνα απεικονίζει το γραφικό περιβάλλον διαμόρφωσης θέσεων του ρομποτικού
βραχίονα (Robot Position Configurator). Μέσα από αυτό το περιβάλλον, ο χρήστης μπορεί να
αποθηκεύσει θέσεις σύλληψης, ενδιάμεσες θέσεις και θέσεις απελευθέρωσης αντικειμένων,
χρησιμοποιώντας free motion. Επιπλέον, παρέχονται επιλογές για έλεγχο των αποθηκευμένων
θέσεων και αυτόματη δημιουργία ενδιάμεσων κινήσεων.

% Νίκος
\section{2ο Μέρος (β)- Color Based}


\end{document}